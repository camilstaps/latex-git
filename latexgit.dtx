% \iffalse meta-comment
%<*internal>
\def\nameofplainTeX{plain}
\ifx\fmtname\nameofplainTex\else
  \expandafter\begingroup
\fi
%</internal>
%<*install>
\input docstrip.tex
\keepsilent
\askforoverwritefalse
\preamble
latexgit
Author: Camil Staps <info@camilstaps.nl>
\endpreamble
\postamble
Copyright (c) 2016 Camil Staps <info@camilstaps.nl>
\endpostamble
\usedir{tex/latex/latexgit}
\generate{
  \file{\jobname.sty}{\from{\jobname.dtx}{package}}
}
%</install>
%<install>\endbatchfile
%<*internal>
\usedir{source/latex/latexgit}
\generate{
    \file{\jobname.ins}{\from{\jobname.dtx}{install}}
}
\nopreamble\nopostamble
\ifx\fmtname\nameofplainTeX
  \expandafter\endbatchfile
\else
  \expandafter\endgroup
\fi
%</internal>
%<*driver>
\documentclass{ltxdoc}
\usepackage[T1]{fontenc}
\usepackage{lmodern}
\usepackage{\jobname}
\usepackage[numbered]{hypdoc}
\usepackage{hyperref}
\usepackage{enumitem}
\EnableCrossrefs
\CodelineIndex
\RecordChanges
\newlist{options}{description}{1}
\setlist[options]{font=\small\texttt,itemsep=0pt}
\newcommand{\git}[0]{\textsf{git}}
\newcommand{\gitcmd}[1]{\texttt{\small#1}}
\newcommand{\gitopt}[1]{%
  \texttt{\small\hyperref[options]{\textcolor{black}{#1}}}%
}
\def\GitOptions#1,#2\relax{%
  \gitopt{#1}%
  \begingroup%
  \ifx\relax#2\relax %
    \def\next{\endgroup}%
  \else%
    \def\next{, \endgroup\GitOptions#2\relax}%
  \fi%
  \next%
}
\def\gitoptions#1{%
  \par
  Recognised options:
  \begingroup%
  \edef\@tempa{#1,}%
  \expandafter\endgroup%
  \expandafter\GitOptions\@tempa\relax%
  \unskip.
}
\begin{document}
  \DocInput{\jobname.dtx}
  \newpage
  \PrintChanges
  \newpage
  \PrintIndex
\end{document}
%</driver>
% \fi
%
%\GetFileInfo{\jobname.sty}
%
%\title{The \latexgit{} package}
%\author{Camil Staps\thanks{Email: info@camilstaps.nl}}
%\date{Version \gitcommithash[shortHash]}
%
%\maketitle
%
%\begin{abstract}
%  This is version \gitcommithash[shortHash] of the \latexgit{} documentation.
%  It was committed by \gitcommitauthorname{} on
%  \gitcommitdate[formatDate,formatTime]. The previous commit was
%  \gitcommithash[shortHash,revision=HEAD^]. Changes since then:
%  \gitcommitmsg.
%\end{abstract}
%
%\tableofcontents
%
%\section{Implementation}
%Define the package and load required packages.
%
%    \begin{macrocode}
%<*package>
\NeedsTeXFormat{LaTeX2e}
\ProvidesPackage{latexgit}

\RequirePackage{pgfkeys}
\RequirePackage{datetime}
%    \end{macrocode}
%
%\begin{macro}{\latexgit}
%    Prints the name of the package.
%    \begin{macrocode}
\newcommand{\latexgit}[0]{\LaTeX{}git}
%    \end{macrocode}
%\end{macro}
%
%\begin{macro}{\gitresult}
%    When a \git{} command is run, the result is stored in \cs{git@rawresult}.
%    However, using \cs{read} results in spacing at the end of the macro.
%    Hence, we need to use \cs{unskip} to remove this spacing.
%    This is a wrapper for \cs{git@rawresult} that adds this \cs{unskip}.
%    \begin{macrocode}
\newcommand{\gitresult}[0]{\git@rawresult\unskip}
%    \end{macrocode}
%\end{macro}
%
%    We now define the keys accepted by git macros. The following options are
%    recognised. Check the documentation of the macro to see which options are
%    considered.
%
%    \begin{options}
%      \label{options}
%      \item[directory=DIR] use the git repository in directory \verb$DIR$ (or
%        its first parent directory that is a git repository).
%      \item[formatDate] format dates using \textsf{datetime}'s
%        \cs{formatdate}.
%      \item[formatTime] format times using \textsf{datetime}'s
%        \cs{formattime}.
%      \item[formatInterDateTime] if both \gitopt{formatDate} and
%        \gitopt{formatTime} are set, this is put in between (e.g.
%        `\verb$\space{}at\space{}$' in English --- this is also the default).
%      \item[revision=REV] get the hash of revision \verb$REV$ (e.g.
%        \gitcmd{master}, \gitcmd{HEAD\textasciicircum}, etc.).
%      \item[showTimeZone] when \gitopt{formatTime} is set: show the time zone
%        between parentheses after the time.
%      \item[shortHash] get only the first seven characters of the hash, as in
%        \gitcmd{git log -{}-oneline}.
%    \end{options}
%
%    \begin{macrocode}
\newif\ifgit@opt@FormatDate
\newif\ifgit@opt@FormatTime
\newif\ifgit@opt@ShortHash
\newif\ifgit@opt@ShowTimeZone
\pgfkeys{
  /git/.is family, /git,
  default/.style={
    directory=.,
    formatDate=false,
    formatInterDateTime=\space{}at\space{},
    formatTime=false,
    revision=HEAD,
    showTimeZone=false,
    shortHash=false,
  },
  directory/.estore in=\git@opt@Directory,
  formatDate/.is if=git@opt@FormatDate,
  formatInterDateTime/.estore in=\git@opt@FormatInterDateTime,
  formatTime/.is if=git@opt@FormatTime,
  revision/.estore in=\git@opt@Revision,
  showTimeZone/.is if=git@opt@ShowTimeZone,
  shortHash/.is if=git@opt@ShortHash,
}
%    \end{macrocode}
%
%\begin{macro}{\git@command}
%    Run a \git{} command and read the output into \cs{git@rawresult}. Before
%    running the command, the directory will change to \cs{git@opt@Directory}.
%    \begin{macrocode}
\newcommand{\git@command}[1]{%
  \newread\git@stream%
  \def\git@rawresult{}%
  \openin\git@stream|"cd \git@opt@Directory; #1"
  \ifeof\git@stream%
    \PackageError{latexgit}%
      {invoke LaTeX with the -shell-escape flag}%
      {invoke LaTeX with the -shell-escape flag}%
  \else%
    \read\git@stream to \git@streamoutput%
    \global\let\git@rawresult\git@streamoutput%
  \fi%
}
%    \end{macrocode}
%\end{macro}
%
%\begin{macro}{\GitCommitHash}
%    Get a commit hash.
%    \gitoptions{revision,shortHash}
%
%    \begin{macrocode}
\newcommand{\gitcommithash}[1][]{\GitCommitHash[#1]\gitresult}
\newcommand{\GitCommitHash}[1][]{%
  \pgfkeys{/git,default,#1}%
  \ifgit@opt@ShortHash%
    \git@command{git log -n 1 --oneline \git@opt@Revision | cut -d' ' -f1}%
  \else%
    \git@command{git log -n 1 \git@opt@Revision | head -1 | cut -d' ' -f2}%
  \fi%
}
%    \end{macrocode}
%\end{macro}
%
%\begin{macro}{\GitCommitMsg}
%    Get a commit message.
%    \gitoptions{revision}
%
%    \begin{macrocode}
\newcommand{\gitcommitmsg}[1][]{\GitCommitMsg[#1]\gitresult}
\newcommand{\GitCommitMsg}[1][]{%
  \pgfkeys{/git,default,#1}%
  \git@command{git log -n 1 --oneline \git@opt@Revision | cut -d' ' -f2-}%
}
%    \end{macrocode}
%\end{macro}
%
%\begin{macro}{\git@formatCommitDate}
%    Format a commit date in ISO format using \textsf{datetime}'s
%    \cs{formatdate}.
%    \begin{macrocode}
\def\git@formatCommitDate#1-#2-#3 #4:#5:#6 +#7\relax{%
  \formatdate{#3}{#2}{#1}%
}
%    \end{macrocode}
%\end{macro}
%
%\begin{macro}{\git@formatCommitTime}
%    Format a commit time using \textsf{datetime}'s \cs{formattime}.
%    \gitoptions{showTimeZone}
%    \begin{macrocode}
\def\git@formatCommitTime#1-#2-#3 #4:#5:#6 +#7\relax{%
  \formattime{#4}{#5}{#6}%
  \ifgit@opt@ShowTimeZone%
    \space(+#7\unskip)%
  \fi%
}
%    \end{macrocode}
%\end{macro}
%
%\begin{macro}{\GitCommitDate}
%    Get a commit date. If \gitopt{formatDate} is set,
%    \cs{git@formatCommitDate} will be used. If \gitopt{formatTime} is set, and
%    \gitopt{formatDate} is unset, \cs{git@formatCommitTime} will be used.
%    \gitoptions{formatDate,formatTime,revision,showTimeZone}
%
%    \begin{macrocode}
\newcommand{\gitcommitdate}[1][]{%
  \GitCommitDate[#1]%
  \ifgit@opt@FormatDate%
    \expandafter\git@formatCommitDate\git@rawresult\relax%
    \ifgit@opt@FormatTime%
      \git@opt@FormatInterDateTime%
      \expandafter\git@formatCommitTime\git@rawresult\relax%
    \fi
  \else\ifgit@opt@FormatTime%
    \expandafter\git@formatCommitTime\git@rawresult\relax%
  \else
    \gitresult%
  \fi\fi%
}
\newcommand{\GitCommitDate}[1][]{%
  \pgfkeys{/git,default,#1}%
  \git@command{git log -n 1 --date=iso \git@opt@Revision |
    grep Date | head -1 | cut -d' ' -f2-}%
}
%    \end{macrocode}
%\end{macro}
%
%\begin{macro}{\GitCommitAuthor}
%    Get a commit author.
%    \gitoptions{revision}
%
%    \begin{macrocode}
\newcommand{\gitcommitauthor}[1][]{\GitCommitAuthor[#1]\gitresult}
\newcommand{\GitCommitAuthor}[1][]{%
  \pgfkeys{/git,default,#1}%
  \git@command{git log -n 1 \git@opt@Revision |
    grep Author | head -1 | cut -d' ' -f2-}%
}
%    \end{macrocode}
%\end{macro}
%
%\begin{macro}{\GitCommitAuthorName}
%    Get a commit author's name.
%    \gitoptions{revision}
%
%    \begin{macrocode}
\newcommand{\gitcommitauthorname}[1][]{\GitCommitAuthorName[#1]\gitresult}
\newcommand{\GitCommitAuthorName}[1][]{%
  \pgfkeys{/git,default,#1}%
  \git@command{git log -n 1 \git@opt@Revision |
    grep Author | head -1 | cut -d' ' -f2- | cut -d'<' -f1}%
}
%    \end{macrocode}
%\end{macro}
%
%\begin{macro}{\GitCommitAuthorEmail}
%    Get a commit author's email address.
%    \gitoptions{revision}
%
%    \begin{macrocode}
\newcommand{\gitcommitauthoremail}[1][]{\GitCommitAuthorEmail[#1]\gitresult}
\newcommand{\GitCommitAuthorEmail}[1][]{%
  \pgfkeys{/git,default,#1}%
  \git@command{git log -n 1 \git@opt@Revision |
    grep Author | head -1 | cut -d' ' -f2- | cut -d'<' -f2 | cut -d'>' -f1}%
}
%    \end{macrocode}
%\end{macro}
%
%    \begin{macrocode}
%</package>
%    \end{macrocode}
%
%\Finale
